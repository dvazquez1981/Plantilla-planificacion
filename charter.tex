\documentclass[
11pt, % The default document font size, options: 10pt, 11pt, 12pt
%codirector, % Uncomment to add a codirector to the title page
]{charter} 
\usepackage[utf8]{inputenc}   % Paquete para codificación
\usepackage[T1]{fontenc}      % Paquete para fuentes
\usepackage[spanish]{babel}   % Paquete para idioma español
\usepackage{graphicx}         % Para imágenes
\usepackage{tabularx}    
\usepackage{xcolor}   
\usepackage{pdflscape}

\usepackage{geometry}
\usepackage{xcolor}
\usepackage{caption}  



% El títulos de la memoria, se usa en la carátula y se puede usar el cualquier lugar del documento con el comando \ttitle
\titulo{Modernización de contadores de tránsito con comunicación bidireccional} 

% Nombre del posgrado, se usa en la carátula y se puede usar el cualquier lugar del documento con el comando \degreename
%\posgrado{Carrera de Especialización en Sistemas Embebidos} 
\posgrado{Carrera de Especialización en Internet de las Cosas} 
%\posgrado{Carrera de Especialización en Inteligencia Artificial}
%\posgrado{Maestría en Sistemas Embebidos} 
%\posgrado{Maestría en Internet de las cosas}

% Tu nombre, se puede usar el cualquier lugar del documento con el comando \authorname
% IMPORTANTE: no omitir titulaciones ni tildación en los nombres, también se recomienda escribir los nombres completos (tal cual los tienen en su documento)
\autor{Ing. Diego Aníbal Vázquez}

% El nombre del director y co-director, se puede usar el cualquier lugar del documento con el comando \supname y \cosupname y \pertesupname y \pertecosupname
\director{-}
\pertenenciaDirector{-} 
\codirector{} % para que aparezca en la portada se debe descomentar la opción codirector en los parámetros de documentclass
\pertenenciaCoDirector{FIUBA}

% Nombre del cliente, quien va a aprobar los resultados del proyecto, se puede usar con el comando \clientename y \empclientename
\cliente{Subgerencia de Estudios de Demanda}
\empresaCliente{Vialidad Nacional}
 
\fechaINICIO{29 de abril de 2025}		%Fecha de inicio de la cursada de GdP \fechaInicioName
\fechaFINALPlan{17 de junio de 2025} 	%Fecha de final de cursada de GdP
\fechaFINALTrabajo{marzo de 2026}	%Fecha de defensa pública del trabajo final


\begin{document}

\maketitle
\thispagestyle{empty}
\pagebreak


\thispagestyle{empty}
{\setlength{\parskip}{0pt}
\tableofcontents{}
}
\pagebreak


\section*{Registros de cambios}
\label{sec:registro}


\begin{table}[ht]
\label{tab:registro}
\centering
\begin{tabularx}{\linewidth}{@{}|c|X|c|@{}}
\hline
\rowcolor[HTML]{C0C0C0} 
Revisión & \multicolumn{1}{c|}{\cellcolor[HTML]{C0C0C0}Detalles de los cambios realizados} & Fecha      \\ \hline
0      & Creación del documento                                 &\fechaInicioName \\ \hline
1      & Se completa hasta el punto 5 inclusive                & {13} de {mayo} de 2025 \\ \hline
2      & Se completa hasta el punto 9 inclusive        & {20} de {mayo} de 2025 \\
\hline
3      & Se completa hasta el punto 12 inclusive                & {27} de {mayo} de 2025 \\ 
\hline
%		  Se puede agregar algo más \newline
%		  En distintas líneas \newline
%		  Así                                                    & {día} de {mes} de 202X \\ 3      & Se completa hasta el punto 12 inclusive                & {día} de {mes} de 202X \\ \hline
4      & Se completa el plan	                                 & {3} de {junio} de 2025 \\ \hline

% Si hay más correcciones pasada la versión 4 también se deben especificar acá

\end{tabularx}
\end{table}

\pagebreak



\section*{Acta de constitución del proyecto}
\label{sec:acta}

\begin{flushright}
Buenos Aires, \fechaInicioName
\end{flushright}

\vspace{2cm}
Por medio de la presente se acuerda con el \authorname\hspace{1px} que su Trabajo Final de la \degreename\hspace{1px} se titulará ``\ttitle'' y consistirá en {un sistema de comunicación con los contadores de tránsito, que incorpore un modelo bidireccional. El trabajo tendrá un presupuesto preliminar estimado de 600} horas y un costo estimado de  \$ 20.720.000, con fecha de inicio el \fechaInicioName\hspace{1px} y fecha de presentación pública en \fechaFinalName.
Se adjunta a esta acta la planificación inicial.
\vfill
% Esta parte se construye sola con la información que hayan cargado en el preámbulo del documento y no debe modificarla
\begin{table}[ht]
\centering
\begin{tabular}{ccc}
\begin{tabular}[c]{@{}c@{}}Dr. Ing. Ariel Lutenberg \\ Director posgrado FIUBA\end{tabular} & \hspace{2cm} & \begin{tabular}[c]{@{}c@{}}\clientename \\ \empclientename \end{tabular} \vspace{2.5cm} \\ 
\multicolumn{3}{c}{\begin{tabular}[c]{@{}c@{}} \supname \\ Director del Trabajo Final\end{tabular}} \vspace{2.5cm} \\
\end{tabular}
\end{table}

\section{1. Descripción técnica-conceptual del proyecto a realizar}
\label{sec:descripcion}

%\begin{consigna}{red} % ELIMINAR \begin{consigna}{red} y \end{consigna}{red} en las secciones que vayan completando para cada entrega parcial.


\subsection{Contexto y motivación}
Este proyecto surge a partir de una necesidad detectada en la infraestructura actual de monitoreo del tránsito vehicular utilizada en rutas nacionales.

Actualmente, uno de los modelos de contadores de tránsito empleados ha sido desarrollado internamente y cumple adecuadamente su función básica: registrar el paso de vehículos, clasificarlos por carril en livianos y pesados y transmitir los datos a un servidor central.

Sin embargo, estas unidades se comunican exclusivamente mediante enlaces GPRS tercerizados a través de un canal unidireccional. Esto impide cualquier tipo de interacción remota con los dispositivos en campo.

Frente a esta situación, se plantea el desafío de modernizar la arquitectura de comunicaciones del sistema, incorporando capacidades de comunicación bidireccional, diagnóstico remoto y respuesta operativa rápida.

\subsection{Problemas identificados}

\begin{itemize}

\item Falta de comunicación bidireccional: actualmente, no es posible enviar comandos desde el servidor a los dispositivos para ajustar su configuración, reiniciarlos o recolectar información de diagnóstico.

\item Dependencia de proveedores externos: la infraestructura GPRS utilizada es tercerizada, lo que genera costos recurrentes, posibles restricciones técnicas y dificultades para gestionar incidentes de manera eficiente.

\item Imposibilidad de actualización remota: cualquier modificación de parámetros de funcionamiento requiere intervención física en el dispositivo, lo que limita la flexibilidad y agilidad operativa.

\end{itemize}

\subsection{Estado del arte y propuesta de valor}
En el mercado existen diversas soluciones comerciales que ofrecen capacidades de gestión remota y comunicación bidireccional. Sin embargo, muchas de ellas resultan costosas.

El enfoque propuesto busca aprovechar tecnologías de código abierto y protocolos estandarizados (MQTT), con el objetivo de construir una alternativa flexible, escalable y económicamente viable adaptada al entorno específico de las rutas argentinas.
\subsection{Propuesta de modernización}

Se propone rediseñar el sistema de comunicaciones de los contadores de tránsito mediante la incorporación de un modelo de comunicación bidireccional y segura.
Este nuevo esquema permitirá no solo el envío de datos desde los dispositivos hacia el servidor central, sino también la recepción de comandos y actualizaciones de parámetros desde el servidor hacia los dispositivos en el campo.
Además, se prevé la visualización de los datos en tiempo real conforme se transmiten.

\subsection{Descripción funcional y técnica}

El sistema estará compuesto por un dispositivo contador con capacidad de comunicación bidireccional mediante GPRS, que utilice el protocolo MQTT para el envío y recepción de datos. La información capturada será almacenada en una base de datos relacional y visualizada a través de una interfaz web básica alojada en el servidor central. Esta interfaz permitirá el monitoreo en tiempo real así como el envío de comandos remotos, se incluirán funciones de monitoreo de temperatura, nivel de batería y detección de errores de hardware o comunicación.




\subsection{Grado de innovación}
La innovación del proyecto radica en la integración de elementos existentes , como sensores y redes móviles, bajo una arquitectura abierta y centralizada, adaptable y orientada a la gestión inteligente de los datos de tránsito, con un fuerte énfasis en la autonomía y flexibilidad operativa.


En la figura \ref{fig:diagBloques} se presenta el diagrama en bloques del sistema. El dispositivo de conteo, compuesto por sensores y un microcontrolador, se comunica mediante un canal inalámbrico GPRS y transmite los datos hacia un broker MQTT externo. El servidor central está suscrito a este broker y recibe los datos mediante un módulo de procesamiento que los almacena en una base de datos relacional. Paralelamente, el sistema cuenta con una API REST que permite el acceso a la información desde una interfaz web y la ejecución de comandos administrativos. De esta manera, el sistema combina la eficiencia del protocolo MQTT para el envío de datos en tiempo real con la flexibilidad de una API REST para la gestión y visualización.

\begin{figure}[htpb]
\centering 
\includegraphics[width=.90\textwidth]{./Figuras/diagBloques.png}
\caption{Diagrama en bloques del sistema.}
\label{fig:diagBloques}
\end{figure}

\vspace{25px}


\section{2. Identificación y análisis de los interesados}
\label{sec:interesados}



\begin{table}[h!]
\renewcommand{\arraystretch}{1.4} % Aumenta el espacio entre filas
\centering
\begin{tabular}{|p{2.5cm}|p{3.5cm}|p{3.5cm}|p{3.5cm}|}
\hline
\textbf{Rol} & \textbf{Nombre y apellido} & \textbf{Organización} & \textbf{Puesto} \\
\hline
Auspiciante   & \empclientename             & -                & -                                \\
\hline
Cliente       & \clientename                & \empclientename  & -                                \\
\hline
Impulsor      & Ing. Rogelio Diego González & \empclientename  & Subgerente de Estudios de Demanda \\
\hline
Responsable   & \authorname                 & FIUBA            & Alumno                           \\
\hline
Colaboradora  & Ing. María Clara Cutrone    & \empclientename  & Jefa Sección Análisis Tránsito   \\
\hline
Colaborador   & Ing. Alejandro Di Rosso     & \empclientename  & Jefe Sección Censos de Carga     \\
\hline
Orientador    & \supname                    & \pertesupname    & Director del Trabajo Final       \\
\hline
Opositores    & ATSA                        & -                & -                                \\
\hline
Usuario final & \clientename                & \empclientename  & -                                \\
\hline
Usuario final & Gerencia Ejecutiva de Proyectos y Obras & \empclientename & -             \\
\hline
Usuario final & Consultoras Externas        & -                & -                                \\
\hline
\end{tabular}
\caption{Identificación de los interesados.}
\label{tab:interesados}
\end{table}



\begin{table}[h!]
\renewcommand{\arraystretch}{1.4} % Aumenta el espacio entre filas
\centering
\begin{tabular}{|p{2.5cm}|p{3.5cm}|p{7.5cm}|}
\hline
\textbf{Rol} & \textbf{Nombre y apellido} & \textbf{Observaciones} \\
\hline
Auspiciante & \empclientename & Es exigente con la rendición de gastos. Se deberá tener especial cuidado en este aspecto. \\
\hline
Cliente & \clientename & Valora especialmente la autonomía operativa, la posibilidad de diagnóstico remoto y la reducción de costos operativos. No hay condiciones especiales relacionadas con la propiedad intelectual ni con la confidencialidad. \\
\hline
Colaboradora & Ing. María Clara Cuttrone & Suele pedir licencia debido a una familia extensa. La planificación debe considerar esta situación. \\
\hline
\end{tabular}
\caption{Participantes y consideraciones relevantes}.
\end{table}

\section{3. Propósito del proyecto}
\label{sec:proposito}

Diseñar e implementar un sistema capaz de registrar eventos de tránsito y permitir la transmisión segura de los datos recolectados, incluso ante cortes de conexión. Además, se busca incorporar capacidades de comunicación bidireccional que habiliten diagnósticos remotos y la carga de actualizaciones de parámetros, así como una mejor respuesta ante fallas. Con ello, se pretende mejorar la calidad de la información recolectada en rutas nacionales y facilitar el trabajo de los equipos técnicos responsables del monitoreo.

\section{4. Alcance del proyecto}
\label{sec:alcance}


Se desarrollará un prototipo funcional que permita validar los aspectos clave del rediseño propuesto.
El prototipo incluirá:

\begin{itemize}
	\item Un dispositivo contador actualmente existente, pero incompleto, al que se le incorporará la capacidad de comunicación bidireccional.
	\item Un canal de comunicación GPRS.
	\item Implementación de un protocolo seguro (MQTT) para el envío y recepción de datos y comandos.
	\item Una interfaz básica en el servidor central para la visualización de datos en tiempo real y el envío de instrucciones al dispositivo.
	\item Funciones elementales de monitoreo remoto.
	\item Base de datos relacional en el servidor central para el almacenamiento estructurado de los datos recibidos desde los dispositivos de campo.
	
\end{itemize}

El presente proyecto no incluye el desarrollo ni la implementación de algoritmos de inteligencia artificial para estimar o reconstruir tránsitos no detectados.

\section{5. Supuestos del proyecto}
\label{sec:supuestos}

Para el desarrollo del presente proyecto se supone que:

\begin{itemize}
	\item Se dispone de acceso a la infraestructura actual de los contadores de tránsito ya instalados, así como de la documentación técnica necesaria para su análisis y eventual integración.
	\item Se cuenta con conectividad intermitente por GPRS en los sitios donde se instalará el sistema, lo que permitirá validar el funcionamiento del envío diferido de datos.
	\item Los recursos humanos involucrados  estarán disponibles durante la duración del proyecto en los tiempos y roles previstos.
	\item Los materiales requeridos (hardware de prueba, sensores, dispositivos de comunicación, etc.), estarán disponibles o serán reemplazables por equivalentes funcionales en caso de faltantes.
	\item La incorporación de comunicación bidireccional es técnicamente factible y puede realizarse sin rediseñar completamente el hardware existente.
\end{itemize}

\section{6. Requerimientos}
\label{sec:requerimientos}


\begin{enumerate}
	\item Requerimientos funcionales:
			\begin{enumerate}
			\item El ESP32-C3 debe recibir datos por interfaz RS-232 desde el sistema de detección.
			\item  Cada evento recibido debe ser encolado en memoria RAM según el orden de llegada.
			\item  El ESP32-C3 debe publicar cada mensaje de la cola a un \textit{broker} MQTT remoto usando GPRS.
			\item  El protocolo MQTT debe utilizar QoS 1 o 2 para asegurar la entrega sin duplicación.
			\item  Debe haber control de reintentos ante fallos de conexión sin duplicar mensajes.
			\item  Si no hay conectividad GPRS disponible, los mensajes deben permanecer en la cola en memoria.
			\item  Al llenarse la cola, los mensajes nuevos pueden descartarse (política FIFO).
			\item  El ESP32-C3 debe  suscribirse a un \textit{topic} MQTT para recibir comandos desde el servidor.
			\item  Al recibir un comando válido, el ESP32-C3 debe ejecutar una acción responder OK, error o   devolver con el estado solicitado por el comando.
			\item  La API REST debe suscribirse al mismo \textit{broker} MQTT y recibir todos los eventos publicados.
			\item  La API REST debe poder enviar comandos al ESP32-C3 publicando en el \textit{topic} correspondiente.
			\item  La interfaz web debe permitir visualizar eventos de tránsito recibidos desde la API REST.
			\item  La interfaz web debe permitir enviar comandos al ESP32-C3 a través de la API REST y posibilitar ver el estado.
		\end{enumerate}
		
	\item Requerimientos de documentación:
		\begin{enumerate}
			\item Documentar la estructura de los mensajes RS-232 esperados (formato, delimitadores).
			\item Especificar el topic MQTT usado y los parámetros de conexión (broker, puerto, QoS).
			\item Incluir diagrama de bloques del flujo de eventos: contador → RS-232 → cola → MQTT → API REST.
			\item Incluir diagrama de bloques del flujo de eventos: API REST → MQTT → RS-232 → contador.
			\item  Documentar los comandos remotos disponibles y el formato de sus respuestas.
			\item Documentar la estructura de los \textit{endpoints} REST y sus respuestas.
		\end{enumerate}

	\item Requerimientos de testing:
		\begin{enumerate}
			\item Probar pérdida de conectividad GPRS y reenvío automático posterior.
			\item Verificar que no haya duplicación ni pérdida de eventos con distintos volúmenes de tráfico.
			\item  Verificar saturación de la cola y descarte correcto de eventos.
			\item  Probar recepción de comandos desde el servidor y respuesta correcta.
			\item  Verificar que la API REST consuma correctamente los eventos desde MQTT.
			\item  Probar que la interfaz web reciba eventos en tiempo real o los consulte en el backend.
		\end{enumerate}
	\item Requerimientos de interfaz:
		\begin{enumerate}
			\item La interfaz web debe ser accesible desde cualquier navegador moderno.
			\item La interfaz debe tener una tabla o lista para visualizar los eventos de tránsito.
			\item La interfaz debe permitir enviar comandos a dispositivos mediante un formulario o botón.
		\end{enumerate}

	
	\item Requerimientos de interoperabilidad:
		\begin{enumerate}
			\item  El sistema debe poder comunicarse con un \textit{broker} MQTT externo configurable.
			\item Responder al protocolo MQTT según lo acordado en el backend.
		\end{enumerate}

	\item Requerimientos de seguridad:
		\begin{enumerate}
			\item La conexión MQTT debe incluir autenticación por credenciales.
			\item La API REST debe requerir autenticación para el envío de comandos o consulta de datos.
	 	\end{enumerate}
\end{enumerate}

\section{7. Historias de usuarios (\textit{Product backlog})}
\label{sec:backlog}
Criterio para estimación de \textit{story points}: cada historia de usuario se evalúa en tres
dimensiones usando la secuencia de Fibonacci: 1, 2, 3, 5, 8, 13, 21. La suma de dificultad,
complejidad y riesgo se redondea al número de Fibonacci más cercano.
\begin{enumerate}

\item
Como usuario del sistema\\
quiero ver los eventos de tránsito en una interfaz web\\
para tener monitoreo visual y registro histórico.\\
\textit{Story points:} 8 (complejidad: 3, dificultad: 2, incertidumbre: 1)\\
Prioridad: media

\item 
Como administrador del sistema\\
quiero que los eventos se envíen al servidor vía MQTT sobre GPRS\\
para recibir todos los tránsitos detectados en tiempo real o diferido.\\
\textit{Story points:} 8 (complejidad: 3, dificultad: 2, incertidumbre: 3)\\
Prioridad: alta

\item
Como desarrollador backend\\
quiero que la API REST se suscriba al \textit{broker} MQTT\\
para recibir y procesar los eventos de tránsito.\\
\textit{Story points:} 13 (complejidad: 3, dificultad: 3, incertidumbre: 3)\\
Prioridad: alta

\item 
Como administrador desde la interfaz web\\
quiero enviar comandos al dispositivo a través de la API REST\\
para controlar remotamente los dispositivos en campo.\\
\textit{Story points:} 8 (complejidad: 3, dificultad: 2, incertidumbre: 3)\\
Prioridad: alta

\item 
Como usuario del sistema\\
quiero usar la interfaz web para enviar comandos\\
para realizar chequeos o pruebas de funcionamiento a distancia.\\
\textit{Story points:} 5 (complejidad: 2, dificultad: 2, incertidumbre: 1)\\
Prioridad: media

\item 
Como desarrollador del sistema,\\
quiero que el ESP32-C3 registre cada evento de tránsito recibido y respuesta a comandos por RS-232 en una cola en memoria RAM,\\
para asegurar que los eventos no se pierdan si no hay conexión GPRS disponible en ese momento.\\
\textit{Story points:} 13 (complejidad: 5, dificultad: 2, incertidumbre: 3)\\
Prioridad: alta

\item 
Como desarrollador del sistema\\
quiero modificar el contador de tránsito en el ESP32-C3 con el objetivo de una correcta interpretación de las tramas recibidas mediante RS-232\
para asegurar que los eventos de tránsito se procesen con precisión y se registren adecuadamente en la cola local, como el manejo de comandos y sus respuestas al servidor central.\\
\textit{Story points:} 8 (complejidad: 3, dificultad: 2, incertidumbre: 1)\\
Prioridad: alta

\item
Como desarrollador del sistema\\
quiero que el ESP32-C3 detecte y maneje errores de comunicación en el puerto RS-232 y en la conexión GPRS y se reconecte automáticamente cuando sea necesario.\\
para garantizar la estabilidad y continuidad del sistema, evitar la pérdida de eventos de tránsito y asegurar una comunicación confiable con el servidor.\\
\textit{Story points:} 5 (complejidad: 3, dificultad: 1, incertidumbre: 1)\\
Prioridad: media

\end{enumerate}

\section{8. Entregables principales del proyecto}
\label{sec:entregables}

\begin{itemize}
	\item Manual de usuario: guía detallada para operadores y administradores sobre el uso de la interfaz web, incluyendo la visualización de eventos de tránsito y el envío de comandos remotos.
	\item Manual técnico de instalación y configuración: instrucciones para la instalación física de los dispositivos ESP32-C3, configuración de la comunicación RS-232 y GPRS, y puesta en marcha del sistema.
	\item Código fuente del firmware (ESP32-C3) que incluirá:
	\begin{itemize}
	\item Recepción y procesamiento de datos a través de RS-232.
	\item Manejo de errores de comunicación y reconexión automática.
	\item Registro de eventos en cola local para garantizar la integridad de los datos.
	\item Envío de datos al servidor mediante MQTT sobre GPRS.
	\item Recepción de comandos mediante MQTT sobre GPRS y respuesta
	\end{itemize}

\item Código fuente del backend (API REST) que incluirá:

\begin{itemize}
\item Suscripción al \textit{broker} MQTT para recibir eventos de tránsito.
\item Procesamiento y almacenamiento de los datos recibidos.
\item Envío de comandos a los dispositivos en campo. 
\end{itemize}

\item Código fuente del frontend que incluirá:
\begin{itemize}
\item Visualización en tiempo real de los eventos de tránsito.
\item Interfaz para el envío de comandos y control remoto de dispositivos.
\end{itemize}

\item Diagramas de arquitectura del sistema:
\begin{itemize}
\item La arquitectura general del sistema.
\item La comunicación entre componentes (contador, ESP32-C3, servidor, interfaz web).
\item El flujo de datos desde la detección de eventos hasta su visualización.
\item El flujo de datos desde el servidor hasta el contador.
\end{itemize}

\item Esquemas eléctricos y de comunicación incluyendo los diagramas de conexiones entre los dispositivos.
\item Plan de pruebas y resultados
\item Memoria del trabajo final

\end{itemize}


\section{9. Desglose del trabajo en tareas}
\label{sec:wbs}

\begin{enumerate}
\item Análisis y diseño del sistema (65 h):
\begin{enumerate}
\item Revisión de requerimientos funcionales y no funcionales (10 h)
\item Diseño de arquitectura del sistema (15 h)
\item Selección y evaluación de tecnologías (15 h)
\item Elaboración de diagramas de flujo y casos de uso (10 h)
\item Revisión y validación del diseño con los interesados (15 h)
\end{enumerate}

\item Aprendizaje en tecnologías (45 h):
\begin{enumerate}
\item Estudio del protocolo MQTT y su implementación (10 h)
\item Estudio ESP32-C3 (10 h)
\item Aprendizaje sobre implementación de API REST (10 h)
\item Capacitación en desarrollo de interfaces web (10 h)
\item Investigación sobre manejo de errores y reconexión en comunicaciones (5 h)
\end{enumerate}

\item Desarrollo del firmware para ESP32-C3 (90 h):
\begin{enumerate}
\item Implementación de envío y recepción de datos por RS-232 (15 h)
\item Gestión de cola en memoria RAM para eventos (15 h)
\item Publicación de eventos en \textit{broker} MQTT sobre GPRS (20 h)
\item Suscripción a tópicos MQTT para recepción de comandos (15 h)
\item Ejecución y respuesta a comandos recibidos (15 h)
\item Manejo de errores de comunicación y reconexión automática (10 h)
\end{enumerate}
\item Testing del firmware para ESP32-C3 (20 h):
\begin{enumerate}
\item Pruebas unitarias de módulos individuales (10 h)
\item Pruebas de integración con módulos de comunicación (10 h)
\end{enumerate}
\item Documentación del firmware para ESP32-C3 (10 h):
\begin{enumerate}
\item Documentación del código fuente y comentarios en línea (5 h)
\item Elaboración de guía de instalación y configuración (5 h)
\end{enumerate}
\item Desarrollo de la API REST (75 h):
\begin{enumerate}
  \item Diseño del modelo entidad-relación y esquema de base de datos (10 h)
  \item Diseño de \textit{endpoints} para recepción de eventos desde MQTT (12 h)
  \item Implementación de lógica para procesamiento de eventos (12 h)
  \item Desarrollo de \textit{endpoints} para envío de comandos al ESP32-C3 (12 h)
  \item Integración con el  \textit{broker} MQTT (12 h)
  \item Implementación de autenticación y seguridad (10 h)
  \item Pruebas unitarias y de integración de la API (7 h)
\end{enumerate}
\item Testing de la API REST (15 h):
\begin{enumerate}
\item Pruebas de carga y rendimiento (5 h)
\item Pruebas de seguridad y validación de datos (5 h)
\item Pruebas de compatibilidad con diferentes usuarios (5 h)
\end{enumerate}
\item Documentación de la API REST (10 h):
\begin{enumerate}
\item Documentación de  \textit{endpoints} y ejemplos de uso (5 h)
\item Elaboración de guía de integración para desarrolladores (5 h)
\end{enumerate}
\item Desarrollo de la interfaz web (65 h):
\begin{enumerate}
\item Diseño de la interfaz de usuario (15 h)
\item Implementación de visualización de eventos de tránsito (15 h)
\item Desarrollo de funcionalidad para envío de comandos (15 h)
\item Integración con la API REST (10 h)
\item Pruebas de usabilidad y funcionalidad (10 h)
\end{enumerate}
\item Testing de la interfaz web (15 h):
\begin{enumerate}
\item Pruebas de compatibilidad con diferentes navegadores (5 h)
\item Pruebas de accesibilidad y experiencia de usuario (5 h)
\item Pruebas de rendimiento y tiempos de carga (5 h)
\end{enumerate}
\item Documentación de la interfaz web (10 h):
\begin{enumerate}
\item Manual de usuario y guía de navegación (5 h)
\item Documentación de instalación y despliegue (5 h)
\end{enumerate}
\item Infraestructura y configuración del  \textit{broker} MQTT (25 h):
\begin{enumerate}
\item Instalación y configuración del  \textit{broker} MQTT (10 h)
\item Configuración de tópicos y políticas de QoS (10 h)
\item Pruebas de comunicación entre dispositivos y  \textit{broker} (5 h)
\end{enumerate}
\item Documentación de infraestructura (10 h):
\begin{enumerate}
\item Documentación de configuración del  \textit{broker} MQTT (5 h)
\item Guía de mantenimiento y monitoreo (5 h)
\end{enumerate}
\item Gestión del proyecto y coordinación (45 h):
\begin{enumerate}
\item Planificación y seguimiento del proyecto (15 h)
\item Reuniones de coordinación y revisión (15 h)
\item Gestión de riesgos y control de calidad (10 h)
\item Comunicación con los interesados y reporte de avances (5 h)
\end{enumerate}
\item Pruebas e integración del sistema (45 h):
\begin{enumerate}
\item Pruebas de integración entre módulos (15 h)
\item Pruebas de sistema y validación con casos de uso (15 h)
\item Pruebas de aceptación por parte del cliente (15 h)
\end{enumerate}
\item Despliegue y mantenimiento inicial (45 h):
\begin{enumerate}
\item Despliegue del sistema en entorno de producción (15 h)
\item Monitoreo y soporte después de despliegue (15 h)
\item Corrección de errores y ajustes finales (15 h)
\end{enumerate}
\item Buffer para imprevistos (60 h):
\begin{enumerate}
\item Tiempo reservado para manejar retrasos, cambios de alcance o problemas técnicos imprevistos.
\end{enumerate}
\end{enumerate}

\textbf{Cantidad total de horas: 600 }

\newpage

\section{10. Diagrama de Activity On Node}
\label{sec:AoN}
En la figura 2 se observa el diagrama AON, que presenta las actividades principales del proyecto y sus dependencias temporales. Las tareas se organizan en forma de red, donde las flechas indican la secuencia de ejecución. El camino crítico (resaltado en rojo) muestra las actividades cuya duración afecta directamente el plazo total del proyecto, por lo que deben gestionarse con especial atención.
\vspace{2cm}
\begin{figure}[htpb]
\centering 
\includegraphics[width=1.1\textwidth]{./Figuras/AoN.png}
\caption{Diagrama de \textit{Activity on Node}.}
\label{fig:AoN}
   
\vspace{0.5em}
{\small \textcolor{red}{El camino crítico está marcado en rojo.}}
\end{figure}

\newpage

\section{11. Diagrama de Gantt}
\label{sec:gantt}
En la figura 3 se presenta el diagrama de Gantt correspondiente a la planificación del proyecto. En él se detallan las tareas a realizar, sus fechas de inicio y fin, así como la duración estimada de cada una. 



\begin{figure}[htpb]
  \begin{center}
    \begin{ganttchart}[
      time slot unit=day,
      time slot format=isodate,
      x unit=0.120cm,
      y unit title=0.6cm,
      y unit chart=0.5cm,
      milestone label font=\scriptsize,
      group label font=\scriptsize,
      bar label font=\tiny,
      milestone/.append style={xscale=4}
      ]{2025-05-01}{2025-08-11}
      \gantttitlecalendar*{2025-05-01}{2025-08-11}{year} \\
      \gantttitlecalendar*{2025-05-01}{2025-08-11}{month} \\

      % Fase 1: Análisis y Diseño (65 h → 22 días)
      \ganttgroup{1. A. y diseño del sistema}{2025-05-01}{2025-05-22} \\
      \ganttbar{1.1 Requisitos func. y no f.}{2025-05-01}{2025-05-02} \\
      \ganttbar{1.2 Diseño arquitectura}{2025-05-03}{2025-05-05} \\
      \ganttbar{1.3 Evaluación tec.}{2025-05-06}{2025-05-08} \\
      \ganttbar{1.4 Diag. de flujo y casos}{2025-05-09}{2025-05-10} \\
      \ganttbar{1.5 Val. diseño}{2025-05-11}{2025-05-22} \\
      \ganttgroup{14. Gestión y coord.}{2025-05-21}{2025-05-22} \\
    

      % Fase 2: Aprendizaje en Tecnologías (45 h → 15 días)
      \ganttgroup{2. Aprendizaje en tec.}{2025-05-23}{2025-06-06} \\
      \ganttbar{2.1 Cap. firmware}{2025-05-23}{2025-05-28} \\
      \ganttbar{2.2 Cap. API REST}{2025-05-29}{2025-06-02} \\
      \ganttbar{2.3 Cap. interfaz web}{2025-06-03}{2025-06-06} \\
      \ganttgroup{14. Gestión y coord.}{2025-06-05}{2025-06-06} \\
    
      % Fase 3: Desarrollo Firmware (90 h → 30 días)
      \ganttgroup{3. Desa. del firmware}{2025-06-07}{2025-07-06} \\
      \ganttbar{3.1 Progra. básica}{2025-06-07}{2025-06-16} \\
      \ganttbar{3.2 Integ. sensores}{2025-06-17}{2025-06-23} \\
      \ganttbar{3.3 Com. MQTT}{2025-06-24}{2025-06-30} \\
      \ganttbar{3.4 Opt. código}{2025-07-01}{2025-07-06} \\
      \ganttgroup{14. Gestión y coord.}{2025-07-01}{2025-07-06} \\
    
      % Fase 4: Testing Firmware (20 h → 7 días)
      \ganttgroup{4. Testing firmware}{2025-07-07}{2025-07-13} \\
      \ganttbar{4.1 Pruebas uni.}{2025-07-07}{2025-07-10} \\
      \ganttbar{4.2 Pruebas int.}{2025-07-11}{2025-07-13} \\
      \ganttgroup{14. Gestión y coord.}{2025-07-13}{2025-07-13} \\
    
      % Fase 5: Documentación Firmware (10 h → 4 días)
      \ganttgroup{5. Doc. firmware}{2025-07-14}{2025-07-17} \\
      \ganttbar{5.1 Manual usuario}{2025-07-14}{2025-07-15} \\
      \ganttbar{5.2 Manual técnico}{2025-07-16}{2025-07-18} \\
      \ganttgroup{14. Gestión y coord.}{2025-07-18}{2025-07-18} \\
    
      % Fase 6: Desarrollo API REST (75 h → 25 días)
      \ganttgroup{6. Desa. API REST}{2025-07-18}{2025-08-11} \\
      \ganttbar{6.1 Diseño endpoints}{2025-07-18}{2025-07-22} \\
      \ganttbar{6.2 Implem.}{2025-07-23}{2025-08-05} \\
      \ganttbar{6.3 Seguridad y aut.}{2025-08-06}{2025-08-11} \\
      \ganttgroup{14. Gestión y coord.}{2025-08-10}{2025-08-11} \\
    
    \end{ganttchart}
  \end{center}
  \begin{minipage}{\linewidth}
    \caption{\small Diagrama de Gantt parte 1.}
  \end{minipage}
  \label{fig:gantt1}
\end{figure}

% ---------------------

\begin{figure}[htpb]
  \begin{center}
    \begin{ganttchart}[
      time slot unit=day,
      time slot format=isodate,
     x unit=0.18cm,      % antes 0.120cm
y unit title=0.6cm,  % antes 0.6cm
y unit chart=0.5cm, % antes 0.5cm
      milestone label font=\scriptsize,
      group label font=\scriptsize,
      bar label font=\tiny,
      milestone/.append style={xscale=4}
      ]{2025-08-12}{2025-10-23}
      \gantttitlecalendar*{2025-08-12}{2025-10-23}{year} \\
      \gantttitlecalendar*{2025-08-12}{2025-10-23}{month} \\

      % Fase 7: Testing API REST (15 h → 5 días)
      \ganttgroup{7. Testing API REST}{2025-08-12}{2025-08-16} \\
      \ganttbar{7.1 Pruebas f.}{2025-08-12}{2025-08-14} \\
      \ganttbar{7.2 Pruebas de seg.}{2025-08-15}{2025-08-16} \\
      \ganttgroup{14. Gestión y coord.}{2025-08-15}{2025-08-16} \\
      \ganttbar{14.2/3 Seg.y rep.}{2025-08-15}{2025-08-16} \\

      % Fase 8: Documentación API REST (10 h → 4 días)
      \ganttgroup{8. Doc. API REST}{2025-08-17}{2025-08-20} \\
      \ganttbar{8.1 Manual de Uso}{2025-08-17}{2025-08-18} \\
      \ganttbar{8.2 Doc. técnica}{2025-08-19}{2025-08-20} \\
      \ganttgroup{14. Gestión y coord.}{2025-08-19}{2025-08-20} \\
    
      % Fase 9: Desarrollo Interfaz Web (65 h → 22 días)
      \ganttgroup{9. Desa. interfaz web}{2025-08-21}{2025-09-11} \\
      \ganttbar{9.1 Diseño UI/UX}{2025-08-21}{2025-08-25} \\
      \ganttbar{9.2 Implem.}{2025-08-26}{2025-09-05} \\
      \ganttbar{9.3 Pruebas}{2025-09-06}{2025-09-11} \\
      \ganttgroup{14. Gestión y coord.}{2025-09-10}{2025-09-11} \\
    
      % Fase 10: Testing Interfaz Web (15 h → 5 días)
      \ganttgroup{10. Testing interfaz web}{2025-09-12}{2025-09-16} \\
      \ganttbar{10.1 Pruebas func.}{2025-09-12}{2025-09-14} \\
      \ganttbar{10.2 Pruebas usab.}{2025-09-15}{2025-09-16} \\
      \ganttgroup{14. Gestión y coord.}{2025-09-15}{2025-09-16} \\
    
      % Fase 11: Documentación Interfaz Web (10 h → 4 días)
      \ganttgroup{11. Doc. interfaz web}{2025-09-17}{2025-09-20} \\
      \ganttbar{11.1 Manual de Usuario}{2025-09-17}{2025-09-18} \\
      \ganttbar{11.2 Documentación técnica}{2025-09-19}{2025-09-20} \\
      \ganttgroup{14. Gestión y coord.}{2025-09-19}{2025-09-20} \\
    
      % Fase 12: Integración del Sistema (30 h → 10 días)
      \ganttgroup{12. Integración}{2025-09-21}{2025-10-04} \\
      \ganttbar{12.1 Integración firmware-API}{2025-09-21}{2025-09-26} \\
      \ganttbar{12.2 Integración API-UI}{2025-09-27}{2025-10-04} \\
      \ganttgroup{14. Gestión y coord.}{2025-09-30}{2025-10-04} \\
    
      % Fase 13: Pruebas del Sistema (40 h → 14 días)
      \ganttgroup{13. Pruebas del sistema}{2025-10-05}{2025-10-18} \\
      \ganttbar{13.1 Pruebas integrales}{2025-10-05}{2025-10-12} \\
      \ganttbar{13.2 Pruebas de usuario}{2025-10-13}{2025-10-18} \\
      \ganttgroup{14. Gestión y coord.}{2025-10-17}{2025-10-18} \\
    
      % Fase 15: Cierre del Proyecto (15 h → 5 días)
      \ganttgroup{15. Cierre proyecto}{2025-10-19}{2025-10-23} \\
      \ganttbar{15.1 Informe final}{2025-10-19}{2025-10-22} \\
      \ganttbar{15.2 Presentación}{2025-10-23}{2025-10-23} \\
      \ganttgroup{14. Gestión y coord.}{2025-10-23}{2025-10-23} \\
    
    \end{ganttchart}
  \end{center}
  \begin{minipage}{\linewidth}
    \caption{\small Diagrama de Gantt parte 2.}
  \end{minipage}
  \label{fig:gantt2}
\end{figure}
\newpage
Supuestos considerados para la planificación:
\begin{itemize}
    \item La jornada de trabajo estimada es de 4 horas diarias, excluyendo fines de semana.
    \item Las tareas se desarrollan mayormente de forma secuencial, dado que se cuenta con un único recurso humano.
    \item Las actividades vinculadas a la gestión del proyecto se realizan en paralelo al desarrollo técnico.
\end{itemize}

\section{12. Presupuesto detallado del proyecto}
\label{sec:presupuesto}

\begin{table}[htpb]
\centering
\begin{tabularx}{\linewidth}{@{}|X|c|r|r|@{}}
\hline
\rowcolor[HTML]{C0C0C0} 
\multicolumn{4}{|c|}{\cellcolor[HTML]{C0C0C0}COSTOS DIRECTOS} \\ \hline
\rowcolor[HTML]{C0C0C0} 
Descripción &
  \multicolumn{1}{c|}{\cellcolor[HTML]{C0C0C0}Cantidad} &
  \multicolumn{1}{c|}{\cellcolor[HTML]{C0C0C0}Valor unitario} &
  \multicolumn{1}{c|}{\cellcolor[HTML]{C0C0C0}Valor total} \\ \hline

 Horas de desarrollo del responsable &
  \multicolumn{1}{c|}{600} &
  \multicolumn{1}{c|}{\$18.000 } &
  \multicolumn{1}{c|}{\$10.800.000} \\ \hline
 PostGrado IOT&
  \multicolumn{1}{c|}{1} &
  \multicolumn{1}{c|}{\$4.000.000} &
  \multicolumn{1}{c|}{\$4.000.000} \\ \hline
\multicolumn{3}{|c|}{SUBTOTAL} &
  \multicolumn{1}{c|}{\$14.800.000} \\ \hline
\rowcolor[HTML]{C0C0C0} 
\multicolumn{4}{|c|}{\cellcolor[HTML]{C0C0C0}COSTOS INDIRECTOS} \\ \hline
\rowcolor[HTML]{C0C0C0} 
Descripción &
  \multicolumn{1}{c|}{\cellcolor[HTML]{C0C0C0}Cantidad} &
  \multicolumn{1}{c|}{\cellcolor[HTML]{C0C0C0}Valor unitario} &
  \multicolumn{1}{c|}{\cellcolor[HTML]{C0C0C0}Valor total} \\ \hline

 40 por ciento de gastos directos &
  \multicolumn{1}{c|}{1} &
  \multicolumn{1}{c|}{\$5.920.000 } &
  \multicolumn{1}{c|}{\$5.920.000} \\ \hline


\multicolumn{1}{|l|}{} &
   &
   &
   \\ \hline
\multicolumn{1}{|l|}{} &
   &
   &
   \\ \hline
\multicolumn{1}{|l|}{} &
   &
   &
   \\ \hline
\multicolumn{3}{|c|}{SUBTOTAL} &
  \multicolumn{1}{c|}{\$5.920.000} \\ \hline
\rowcolor[HTML]{C0C0C0}
\multicolumn{3}{|c|}{TOTAL} & \$20.720.000
   \\ \hline
\end{tabularx}%
\end{table}

Los valores de los precios están en pesos argentinos.


\section{13. Gestión de riesgos}

\label{sec:riesgos}

a) Identificación de Riesgos y Consecuencias  \\
\\
Riesgo 1: incompatibilidad del módulo SIM800L con comunicación MQTT
\begin{itemize}
\item Severidad (S): 8\\
Justificación: requeriría rediseñar todo el sistema de comunicación, impactando en el cronograma.
\item Probabilidad (O): 6\\
Justificación: el SIM800L tiene limitaciones conocidas para implementar MQTT de forma estable.
\end{itemize}

Riesgo 2: fallas prolongadas en la conexión GPRS
\begin{itemize}
\item Severidad (S): 7\\
Justificación: pérdida de datos críticos de tránsito durante los periodos sin conexión.
\item Probabilidad (O): 7\\
Justificación: Las redes celulares en zonas rurales suelen tener intermitencias frecuentes.
\end{itemize}

Riesgo 3: saturación de la memoria RAM del ESP32-C3
\begin{itemize}
\item Severidad (S): 6\\
Justificación: pérdida de datos no transmitidos.
\item Probabilidad (O): 5\\
Justificación: depende del volumen de tráfico y frecuencia de eventos.
\end{itemize}

Riesgo 4: vulnerabilidades de seguridad en la comunicación
\begin{itemize}
\item Severidad (S): 8\\
Justificación: podría permitir acceso no autorizado o manipulación de datos.
\item Probabilidad (O): 4\\
Justificación: el uso de protocolos seguros reduce este riesgo.
\end{itemize}

Riesgo 5: incompatibilidad del protocolo RS232 con el contador desarrollado por María Clara Cutrone
\begin{itemize}
\item Severidad dependencia de desarrollo externo y posible incompatibilidad o falta de documentación.
\item Probabilidad (O): 5\\
Justificación: riesgo moderado dado que el protocolo RS232 es estándar, pero la implementación específica puede variar.
\end{itemize}

b) Tabla de Gestión de Riesgos

\begin{table}[htpb]
\centering
\begin{tabularx}{\linewidth}{@{}|X|c|c|c|c|c|c|@{}}
\hline
\rowcolor[HTML]{C0C0C0} 
Riesgo & S & O & RPN & S* & O* & RPN* \\ \hline
Incompatibilidad del módulo SIM800L con comunicación MQTT & 8 & 6 & 48 & 5 & 3 & 15 \\ \hline
Fallas prolongadas en la conexión GPRS & 7 & 7 & 49 & 5 & 4 & 20 \\ \hline
Saturación de la memoria RAM del ESP32-C3 & 6 & 5 & 30 & 5 & 3 & 15 \\ \hline
Vulnerabilidades de seguridad en la comunicación & 8 & 4 & 32 & 6 & 2 & 12 \\ \hline
Incompatibilidad del protocolo RS232 con el contador & 7 & 5 & 35 & 5 & 3 & 15 \\ \hline
\end{tabularx}%
\end{table}

Criterio adoptado: \\
Se tomarán medidas de mitigación en riesgos con RPN > 30.

c) Plan de Mitigación
Riesgo 1: incompatibilidad SIM800L/MQTT
\begin{itemize}
\item Mitigación: implementar protocolo HTTP como alternativa, usando POST para enviar datos.
\item S: 5 (impacto reducido al tener alternativa funcional)
\item O: 3 (solución probada y documentada)
\end{itemize}

Riesgo 2: fallas conexión GPRS
\begin{itemize}
\item Mitigación: implementar almacenamiento local persistente y lógica de reintentos inteligente.
\item S*: 5 (datos se recuperan post-reconexión)
\item O*: 4 (mejora en manejo de desconexiones)
\end{itemize}

Riesgo 4: vulnerabilidades seguridad
\begin{itemize}
\item Mitigación: implementar autenticación por tokens JWT y encriptación TLS.
\item S: 6 (reduce impacto de posibles ataques)
\item O: 2 (medidas comprobadas reducen vulnerabilidades)
\end{itemize}

Riesgo 5: incompatibilidad protocolo RS232 con contador
\begin{itemize}
\item Mitigación: coordinar con María Clara Cuttrone para validar y documentar la interfaz RS232; realizar pruebas de integración tempranas.
\item S*: 5 (reducción del impacto con mejor conocimiento)
\item O*: 3 (mejora en la probabilidad al estar en contacto con desarrolladora)
\end{itemize}



\section{14. Gestión de la calidad}
\label{sec:calidad}



\begin{itemize} 

\item Req \#1: El ESP32-C3 debe recibir datos por interfaz RS-232 desde el sistema de detección.

\begin{itemize}
	\item Verificación: realizar pruebas de laboratorio con el sistema de detección enviando datos por RS-232 y monitorear que el ESP32-C3 los reciba correctamente sin pérdidas ni errores. Se usan analizadores lógicos y herramientas de debug para observar la comunicación interna.
	\item Validación: demostrar con el cliente la recepción en tiempo real de eventos generados en el sistema de detección y confirmación del correcto procesamiento de datos.
\end{itemize}

\item Req \#3: El ESP32-C3 debe publicar cada mensaje de la cola a un \textit{broker} MQTT remoto usando GPRS.

\begin{itemize}
	\item Verificación: simular eventos y verificar en laboratorio que el ESP32-C3 publique correctamente cada mensaje a un broker MQTT configurado, validando los logs y el orden de envío.
	\item Validación: mostrar al cliente la recepción de mensajes en el sistema remoto bajo diferentes condiciones de conectividad y volumen de datos.
\end{itemize}

\item Req \#4: El protocolo MQTT debe utilizar QoS 1 o 2 para asegurar la entrega sin duplicación.

\begin{itemize}
	\item Verificación: testear la implementación con broker MQTT que soporte QoS 1 y 2, revisando que no haya duplicados en la recepción y confirmando los mecanismos de ACK.
	\item Validación: validar con el cliente que los datos recibidos en el backend no presentan duplicaciones y que la información es consistente.
\end{itemize}

\item Req \#5: Debe haber control de reintentos ante fallos de conexión sin duplicar mensajes.

\begin{itemize}
	\item Verificación: en laboratorio interrumpir la conexión GPRS y observar que el ESP32-C3 hace reintentos controlados, sin duplicar mensajes en el broker.
	\item Validación: el cliente valida que durante periodos de desconexión los datos no se pierden ni duplican, comprobando reportes posteriores.
\end{itemize}

\item Req \#6: Si no hay conectividad GPRS disponible, los mensajes deben permanecer en la cola en memoria.

\begin{itemize}
	\item Verificación: simular falta de conectividad y verificar que los mensajes se almacenan en cola interna y se publican una vez restaurada la conexión.
	\item Validación: el cliente confirma que no se pierden eventos durante desconexiones y que la cola gestiona adecuadamente la memoria.
\end{itemize}

\item Req \#8: El ESP32-C3 debe suscribirse a un \textit{topic} MQTT para recibir comandos desde el servidor.

\begin{itemize}
	\item Verificación: comprobar en pruebas que el ESP32-C3 se suscribe correctamente y recibe comandos publicados en el topic específico.
	\item Validación: demostrar al cliente que los comandos enviados desde el servidor llegan y se interpretan correctamente en el dispositivo.
\end{itemize}

\item Req \#11: La API REST debe poder enviar comandos al ESP32-C3 publicando en el \textit{topic} correspondiente.

\begin{itemize}
	\item Verificación: probar en laboratorio la API REST enviando comandos y confirmar la publicación correcta en MQTT.
	\item Validación: el cliente verifica que los comandos emitidos desde la API llegan efectivamente al dispositivo y se ejecutan.
\end{itemize}

\item Req \#15: Probar recepción de comandos desde el servidor y respuesta correcta.

\begin{itemize}
	\item Verificación: realizar pruebas funcionales enviando comandos y validando que el ESP32-C3 responde con OK, error o estados según corresponda.
	\item Validación: el cliente valida que el sistema responde adecuadamente ante comandos válidos y maneja errores según lo esperado.
\end{itemize}

\item Req \#18: La conexión MQTT debe incluir autenticación por credenciales.

\begin{itemize}
	\item Verificación: revisar configuración de autenticación en el broker MQTT y probar conexiones con credenciales válidas e inválidas.
	\item Validación: el cliente confirma que el acceso está restringido y solo usuarios autorizados pueden conectarse y publicar/suscribirse.
\end{itemize}

\item Req \#19: La API REST debe requerir autenticación para el envío de comandos o consulta de datos.

\begin{itemize}
	\item Verificación: probar mecanismos de autenticación (por ejemplo, tokens o claves) en la API REST, validando que rechaza accesos no autorizados.
	\item Validación: el cliente verifica que solo usuarios autenticados pueden enviar comandos o consultar datos mediante la API.
\end{itemize}

\end{itemize}

\section{15. Procesos de cierre}    
\label{sec:cierre}
\label{sec:reunion-final}

Para la realización de la reunión final de evaluación del proyecto se establecen las siguientes pautas de trabajo:

\begin{itemize}
    \item Análisis del plan de proyecto original: \\
     Responsable: el director del proyecto (profesor a cargo). \\
     Procedimiento: se revisará el cronograma inicial, los entregables planificados y los resultados obtenidos, mediante la comparación con los reportes de avance elaborados durante el desarrollo del proyecto. 

    \item Identificación de técnicas y procedimientos útiles e inútiles, problemas y soluciones: \\
     Responsables: el alumno con los  colaboradores. \\
     Procedimiento: se llevará a cabo una reflexión crítica sobre las herramientas y métodos empleados. Se documentarán los problemas surgidos y las soluciones aplicadas, con el objetivo de extraer lecciones aprendidas para futuras experiencias similares. Estas conclusiones se incluirán en un documento de cierre.

    \item Organización de un acto simbólico de cierre y agradecimiento: \\
    Responsable: El director del proyecto y el alumno. \\
    Procedimiento: se coordinará un breve encuentro ( virtual) con los colaboradores y el director para agradecer su participación. Se enviarán agradecimientos formales y se compartirá un resumen de los resultados obtenidos.
\end{itemize}



\end{document}
